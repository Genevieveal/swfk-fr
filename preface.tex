
% This work is licensed under the Creative Commons Attribution-Share Alike 2.0 France License.
% To view a copy of this license, visit http://creativecommons.org/licenses/by-sa/2.0/fr/legalcode
% or send a letter to Creative Commons, 171 Second Street, Suite 300, San Francisco, California, 94105, USA.



\markboth{}{}

\chapter{Préface}
\section*{\center{\textit{Une note aux parents...}}}

Chers parents et autres personnes attentionnées,\\

\section*{Vous vous demandez peut-être pourquoi apprendre à programmer?}
Apprendre à programmer permettra à votre enfant d'améliorer sa logique. Un ordinateur ne fait que ce qu'on lui a demandé. Si le programme ne fonctionne pas c'est que sa logique interne est mauvaise. 

De plus, savoir comment fonctionnent les ordinateurs permettra à l'enfant de compren\-dre qu'ils ne fonctionnent pas grâce à de la poudre magique mais grâce à la magie du génie humain.


\section*{Vous vous demandez peut-être, pourquoi Python?}
Python est un langage simple mais pas simpliste. Les commandes Python ont des rôles indépendants: «~il doit y avoir une manière évidente, de préférence une seule, de faire les choses~». Ces commandes sont donc en nombre limitées, ce qui permet de se concentrer sur la logique du programme et non pas sur les commandes à utiliser. 
Néanmoins Python est puissant, d'ailleurs des organismes comme l'INRIA ou la NASA utilisent Python. Il est utilisé par des gouvernements pour des infrastructures critiques. Les entreprises l'utilisent comme Google qui fournit d'ailleurs un environnement Python en ligne.

Python est un langage de haut niveau qui ne contient pas de concepts liés au matériel ou au système d'exploitation ce qui permet de réaliser des programmes simples sans se focaliser sur des éléments non directement productifs. Python est interactif, sa ligne de commande permet de réaliser des tests sans passer par des étapes complexes.

Par ailleurs, Python impose une écriture compréhensible car les différents blocs des programmes sont indiqués par les indentations du texte.

\section*{À propos de ce livre.}

Ce livre existe en trois versions: Linux, Mac Os X et Windows. La version que vous avez en main est la version  \begin{WINDOWS}Windows\end{WINDOWS}\begin{LINUX}Linux\end{LINUX}\begin{MAC}Mac Os X\end{MAC}. Si vous n'utilisez pas \begin{WINDOWS}Windows\end{WINDOWS}\begin{LINUX}Linux\end{LINUX}\begin{MAC}Mac Os X\end{MAC}, vous pouvez télécharger la version adaptée sur: \url{http://code.google.com/p/swfk-fr}.
\\
\bigskip
\section*{Comment installer Python?}
De manière à ce que votre enfant puisse commencer à programmer, vous avez besoin d'installer Python sur votre ordinateur. Ce livre a été récemment mis à jour pour Python~3.1; cette version de Python est la plus récente et n'est pas compatible avec les versions antérieures. Si vous avez une version plus ancienne installée, vous devez télécharger la dernière version pour utiliser ce livre.
\\


Installer Python est une tâche assez simple, mais il y a quelques différences selon le système d'exploitation que vous utilisez. 




Si vous venez juste d'acheter un nouvel ordinateur, que vous n'avez pas idée de quoi faire avec et que les phrases précédentes vous ont remplit de frissons glacés, vous devriez sûrement trouver quelqu'un pour faire ça.\\


Selon votre ordinateur et la vitesse de votre connexion à Internet, cette installation devrait vous prendre entre quelques minutes et plusieurs heures.

Premièrement, allez sur \url{www.python.org} et téléchargez le dernier installateur pour Python~3.1. À la date de l'écriture de ce livre vous pouvez le trouver à l'adresse \index{installation} \url{http://www.python.org/ftp/python/3.1/python-3.1.msi}.

Double-cliquez sur l'icône de l'installateur de Python pour Windows (vous-vous rappelez où vous l'avez téléchargé, n'est-ce pas?), et suivez les instructions pour l'installer à l'endroit par défaut (qui est probablement \verb+C:\Python31+ ou quelque chose de très similaire).\\


\textit{Après l'installation...}


... Vous pourriez avoir besoin de vous asseoir à côté de votre enfant pour quelques premiers chapitres, mais 
heureusement après quelques exemples, il devrait chasser vos mains du clavier et faire les choses par lui-même. 
Il devrait, au moins, savoir comment utiliser un éditeur de texte quelconque avant de commencer (non, pas un traitement de texte comme Word, un vrai éditeur de texte à l'ancienne, sans gestion d'effets de style, comme le bloc-notes) il devrait au moins être capable d'ouvrir et de fermer des fichiers, de créer des fichiers «~texte~» et sauvegarder ce qu'il fait. Mis à part ça, ce livre va essayer de lui enseigner le \textit{b+a, ba} à compter de cette page.

\bigskip
Merci pour votre temps, bien cordialement.\\
%\AddToShipoutPicture*{\BackgroundPicture}


\textit{Le Livre}
